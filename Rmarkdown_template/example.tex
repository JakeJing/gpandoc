% Options for packages loaded elsewhere
\PassOptionsToPackage{unicode}{hyperref}
\PassOptionsToPackage{hyphens}{url}
%
\documentclass[
]{article}
\usepackage{lmodern}
\usepackage{amssymb,amsmath,standalone}
\usepackage{ifxetex,ifluatex}
\ifnum 0\ifxetex 1\fi\ifluatex 1\fi=0 % if pdftex
  \usepackage[T1]{fontenc}
  \usepackage[utf8]{inputenc}
  \usepackage{textcomp} % provide euro and other symbols
\else % if luatex or xetex
  \usepackage{unicode-math}
  \defaultfontfeatures{Scale=MatchLowercase}
  \defaultfontfeatures[\rmfamily]{Ligatures=TeX,Scale=1}
\fi
% Use upquote if available, for straight quotes in verbatim environments
\IfFileExists{upquote.sty}{\usepackage{upquote}}{}
\IfFileExists{microtype.sty}{% use microtype if available
  \usepackage[]{microtype}
  \UseMicrotypeSet[protrusion]{basicmath} % disable protrusion for tt fonts
}{}
\makeatletter
\@ifundefined{KOMAClassName}{% if non-KOMA class
  \IfFileExists{parskip.sty}{%
    \usepackage{parskip}
  }{% else
    \setlength{\parindent}{0pt}
    \setlength{\parskip}{6pt plus 2pt minus 1pt}}
}{% if KOMA class
  \KOMAoptions{parskip=half}}
\makeatother
\usepackage{xcolor}
\IfFileExists{xurl.sty}{\usepackage{xurl}}{} % add URL line breaks if available
\IfFileExists{bookmark.sty}{\usepackage{bookmark}}{\usepackage{hyperref}}
\hypersetup{
  pdftitle={My Template for Reformating a Research Article},
  pdfauthor={Yingqi Jing},
  pdfkeywords={Word order typology, language evolution},
  hidelinks,
  pdfcreator={LaTeX via pandoc}}
\urlstyle{same} % disable monospaced font for URLs
\setlength{\emergencystretch}{3em} % prevent overfull lines
\providecommand{\tightlist}{%
  \setlength{\itemsep}{0pt}\setlength{\parskip}{0pt}}
\setcounter{secnumdepth}{5}
\usepackage{mathtools, makecell, tikz}
\usepackage{booktabs}
\usepackage{longtable}
\usepackage{array}
\usepackage{multirow}
\usepackage{wrapfig}
\usepackage{float}
\usepackage{colortbl}
\usepackage{pdflscape}
\usepackage{tabu}
\usepackage{threeparttable}
\usepackage{threeparttablex}
\usepackage[normalem]{ulem}
\usepackage{makecell}
\usepackage{xcolor}
%%%% citation %%%%%
\usepackage{xpatch}
\usepackage[style=authoryear-comp, % compressed form (e.g., Gibson 1998, 2000)
	backend=biber,
	dashed=false, % avoid hide duplicated names in References
	sorting = none,
	maxcitenames=3,
	uniquelist=false, uniquename=false,
	maxbibnames=99]{biblatex}
%\usepackage[]{biblatex}
\addbibresource{/Users/jakejing/switchdrive/bib/references.bib}

%%%% load packages %%%%
\usepackage{gb4e}
\noautomath
\usepackage{standalone}
\usepackage{threeparttablex}
\usepackage{longtable}
\usepackage{amsmath} 
\usepackage{caption}
\usepackage{subcaption}
\usepackage{upgreek}
\usepackage{float}
\usepackage{multirow}



%%%% define dimensions and margins %%%%
\usepackage[letterpaper,margin=1in]{geometry}

%%%% title page %%%%
\usepackage{authblk}
\title{\normalfont\normalsize My Template for Reformating a Research Article}
\frenchspacing
\author[1]{\normalsize Yingqi Jing}
%\author[2, 5]{\normalsize  Dami\'an E. Blasi}
%\author[3]{\normalsize Balthasar Bickel}
\affil[1]{Uppsala University; \href{mailto:yingqi.jing@lingfil.uu.se}{\nolinkurl{yingqi.jing@lingfil.uu.se}}}
%\affil[2]{University of Zurich; dblasi@fas.harvard.edu}
%\affil[3]{University of Zurich; balthasar.bickel@uzh.ch}
%\affil[4]{Uppsala University}
%\affil[5]{Harvard University}
\date{} 

\setcounter{page}{0} 

% Keywords command
\providecommand{\keywords}[1]
{\normalsize	
  \textbf{Keywords:} #1
}

\begin{document}

\maketitle

%%%% abstract %%%%%%
%\clearpage\mbox{}\thispagestyle{empty}\clearpage
\clearpage
\renewcommand{\abstractname}{\normalsize Abstract}
\begin{abstract}
% acknowledgement
\renewcommand*{\thefootnote}{\fnsymbol{footnote}}%
\normalsize This is the abstract of the paper. An abstract summarizes, usually in
one paragraph of 300 words or less, the major aspects of the entire
paper in a prescribed sequence that includes: 1) the overall purpose of
the study and the research problem(s) you investigated; 2) the basic
design of the study; 3) major findings or trends found as a result of
your analysis; and, 4) a brief summary of your interpretations and
conclusions.\footnote{I would like to first thank.}
\setcounter{footnote}{0}
\end{abstract}

\vspace{\baselineskip}
\setlength\parindent{24pt}
\keywords{Word order typology, language evolution}

%{Keywords: Word order typology, language evolution}

%%\renewcommand{\abstractname}{Acknowledgments}
%\begin{abstract}
%I would like to first thank.
%\end{abstract}
%\clearpage\mbox{}\thispagestyle{empty}\clearpage
%% \pagestyle{empty}
%% \thispagestyle{empty}
%% \newpage
%


\clearpage

\hypertarget{introduction}{%
\section{Introduction}\label{introduction}}

You can also load the cached file, \textcite{Greenberg1963} check the
environment to see whether the variables have already been saved
\autocite{Dryer1992}.

In self-paced reading, the verb is read faster in such examples when it
follows the semantically rich and complex noun phrase in
\ref{ex-german}a (``the book that Lisa bought yesterday'') than when it
follows the short object noun phrase ``the book'' in \ref{ex-german}b,
where the relative clause is extraposed \autocite{Konieczny2000}.

\begin{exe} \ex Expectation-based facilitation in German \label{ex-german}
\begin{xlist}
\ex \gll Er hat das Buch, [das Lisa gestern gekauft hatte], hingelegt.\\
         he has the book that Lisa yesterday bought had laid.down\\
\ex  \gll Er hat das Buch hingelegt, [das Lisa gestern gekauft hatte].\\
        he has the book laid.down that Lisa yesterday bought had\\
       \glt ‘He has laid down the book that Lisa had bought yesterday.’ 
\end{xlist}
\end{exe}

\printbibliography[title=References]


\end{document}
